\documentclass[11pt,article,oneside]{memoir} %{{{
% based on Kieran Healy's syllabus templates
% https://github.com/kjhealy/latex-custom-kjh 
\usepackage{booktabs}

\usepackage{org-preamble-pdflatex} 
\usepackage[margin=1.2in]{geometry}

\usepackage{enumitem}
\setlist{nolistsep}

\setlength{\parskip}{10pt}
\setlength{\parindent}{0pt}

%}}}
% Definitions %{{{
\def\myauthor{Author}
\def\mytitle{Title}
\def\mycopyright{\myauthor}
\def\mykeywords{}
\def\mybibliostyle{plain}
\def\mybibliocommand{}
\def\mysubtitle{}
\def\myaffiliation{Indiana University}
\def\myaddress{\url{https://iu.zoom.us/my/yyahn}} 
\def\myemail{yyahn@iu.edu}
\def\myweb{http://yongyeol.com}
\def\myphone{}
\def\myversion{}
\def\myrevision{}

\def\myaffiliation{Indiana University}
\def\myauthor{Yong-Yeol (YY) Ahn}
\def\mykeywords{Visualization, Data, Graduate, Informatics}
\def\mysubtitle{Syllabus}
\def\mytitle{{\normalsize INFO 422 / INFO 590} \\ \HUGE{} Data Visualization}
%\def\mytitle{{\normalsize I 590} \\ \HUGE{} Data Visualization}

%%\chapterstyle{article-3}
%\pagestyle{kjh}

\def\ind{\hangindent=1 true cm\hangafter=1 \noindent}
\def\labelitemi{$\cdot$}

\chapterstyle{article-4}  % alternative styles are defined in latex-custom-kjh/needs-memoir/

%}}}
\begin{document} %{{{

\title{\LARGE \mytitle} %{{{
\author{\Large\myauthor\newline \footnotesize\texttt{\noindent\myemail}}
%\date{Info East 130 (M) / 109 (W)\newline MW 4:00pm--5:15pm. \newline Office hours: W 9:15am-10am}
%\date{Office hours: Monday 2pm-3pm at \myaddress \\(You can send a message anytime on Slack)}

\published{\sffamily Fall 2024 / Mon \& Wed 3:00pm--4:15pm / GA 1118 / Async. Online} 
%\published{\sffamily Fall 2024} 

\maketitle 

\vspace{-20pt}{Office hours \& TAs: see Canvas} \\
{Homepage: \url{https://yyahn.com/dviz-course/}}
%Vincent Wong (\texttt{vmwong@iu.edu}), TBD\\
%Ashutosh Hathidara (\texttt{ashuhath@iu.edu}), Fri. 10am-12pm at \url{https://iu.zoom.us/my/ashuhath}\\\\
%{\bfseries Assistant Instructors (online)}\\  
%Shubham Singh (\texttt{shusingh@iu.edu}), TBD at \url{https://iu.zoom.us/my/shusingh}\\


%}}}
\section{Course Description}%{{{

Data visualization plays a pivotal role in understanding information, from news
articles to cutting-edge scientific research, and is employed across diverse
settings, from home offices to the world's largest corporations. As an integral
component of data analysis, data visualization has become a crucial skill for
all knowledge workers.

This introductory course delves into the core concepts of statistical data
analysis and visualization. You will explore the foundations of data
visualization, covering topics such as perception, integrity, design principles,
statistical methods, data classifications, and various visualization techniques.
Through hands-on exercises utilizing the Python stack, students will develop
practical skills in data processing and visualization.

\paragraph{Relationships with E483/E583 Information Visualization (IVMOOC)} This
course differs from E583/Z637 in that it places greater emphasis on fundamental
statistical visualizations and conducting exploratory data analysis through
coding, utilizing the Python data science and visualization stack. As a result,
it may be a more suitable choice for students aiming for careers in research,
development, engineering, and data analysis, or for those who will directly
handle and analyze complex datasets.

%}}}
\section{Course Objectives}%{{{

Upon completing the course, you are expected to acquire the ability to produce
data visualizations that are effective, accurate, and ethical.  You will also
be able to critically evaluate data visualizations and effectively communicate
your findings to others.  This ability will build on your understanding of the
fundamental principles of data visualization, including human perception,
design, data types, and visualization techniques.  Your proficiency will be
showcased through a course project through which you will not only create data
visualizations but also document the process of creating effective, accurate,
and ethical data visualizations.

%}}}
\section{Communication} %{{{

We will use Canvas and Slack for communication. \textbf{Canvas is for official communications} as well as for anything that contains personal and sensitive information. \textbf{Slack} is for day-to-day information sharing, Q\&As, team discussions, and other casual conversations. 

Announcements, Q\&As, and other communication will be sent via Canvas and Slack. Although the most critical announcements will be sent via both platforms, a lot of course-related information (as well as questions and answers) will be shared on Slack and thus you may miss useful---although not \emph{essential} in completing the course---course-related information if you are not on Slack. 
When joining the course Slack, feel free to avoid using your full name (e.g., use ``John D.'' instead of ``John Doe'') to protect your privacy. 
Also never post your personal information or sensitive data (e.g., grades) to the course Slack. 

The address of the course slack is: \url{https://iu-dviz-course.slack.com}, and visit \url{https://join.slack.com/t/iu-dviz-course/signup} to signup.
You can create an account by using one of the following IU email addresses: \texttt{indiana.edu}, \texttt{umail.iu.edu}, \texttt{iu.edu}, and \texttt{iupui.edu}. Please email the instructor if you want to use other email addresses. 

Please be aware that when seeking answers to questions, using Email and Canvas may result in significantly slower response times than Slack. This is due to instructors being inundated with emails. Anticipate a waiting period of \emph{a few days} (up to a week) for email responses, while Slack messages can generally be expected to receive a reply within a few hours or a day. If your message contains sensitive information, please communicate it via email or Canvas, while sending a message on Slack to notify the instructor that you have sent an email or Canvas message.

Whenever you have something to say about the course or have a suggestion for improving the course, please share your thoughts! We will be extremely grateful if you point out issues in the quizzes, discussions, grading, and so on. You can simply send a message on slack, or anonymously share your opinion:

\url{https://forms.gle/MzzNSV6Y8deJWGC77} 

%}}}
\section{Prerequisites}%{{{
\label{sec:Prerequisites}

Because producing visualizations using Python data \& visualization stack is an integral part of the course, it is required to have a good understanding and working knowledge of programming (esp. Python), as well as working knowledge of using open-source libraries. 
It is also recommended to have a basic understanding of mathematics, statistics, and the Web (HTML, CSS, Javascript, and JSON). 

For self-assessment, visit the following link: \href{http://bit.ly/dvizselfassess}{http://bit.ly/dvizselfassess}. 
Contact the instructor if you are uncertain about your background. 

%}}}
\section{Expectations and Requirements}%{{{
\label{sec:requirements}

%\paragraph{(All sections)} 
%The final assessment will be mainly based on a mid-term exam and a final course project. 
The primary assessment will be through the assignments, the final exam, and the final course project. 
The topic of the final project will be of your (team's) choice, but I encourage everyone to consult with the instructors. 
You are required to submit a final paper that contains not only the \emph{results} but also detailed explanation of the visualization \emph{process} to demonstrate your knowledge on visualization principles and techniques, as well as your ability to apply them to create visualizations. 

% Residential students are expected to attend all course meetings to participate in quizzes and group discussions. If you cannot make it to class due to illness, you should contact the instructor and the AIs \emph{before} class and let us know your situation. We can make accommodations for missed content such as quizzes. These are reviewed on a case-by-case basis.

\paragraph{(Residential course)} You are expected to attend every class and engage in class discussions. 
You are expected to complete all course modules (quizzes and discussions) and assignments, as well as visualization critiques through the ``visualization of the week''. You are also expected to engage in discussion on Canvas and Slack. 
You are not allowed to use your phone or computer during the class unless explicitly asked to do so.  
You are expected to read assigned reading materials prior to the class meetings if there is any.
At the beginning of most class meetings, there will be an \emph{in-class quiz} based on the assigned readings and materials from the previous classes. 
You are expected to complete all weekly assignments. 

\paragraph{(Online)} You are expected to complete all course modules and assignments. 
You are also expected to engage in discussions on Canvas and Slack. 

%}}}
\section{Grading}\label{sec:grading_tentative_}%{{{

I sincerely hope that you focus on your learning and not on the grades! See \url{https://www.youtube.com/watch?v=u6XAPnuFjJc} 

The grade may be curved at the end of the course. Because the gradebook often has ungraded items, \emph{the grade that you can see on Canvas may not be a faithful reflection of your projected grade!}. 

There will be extra credits based on your strong engagement in the course, in terms of sharing useful resources \& interesting visualization-related articles, participating in discussions, and helping other students.

\begin{itemize}%{{{

\item Attendance, Quiz, and Participation: 20\%
%\item Attendance, Quiz, and Participation: 30\% 

\item Assignments: 20\% 

\item Exam: 30\%

\item Final project: 30\%
%\item Final project: 40\%

\end{itemize}%}}}
%}}}
\section{Books and key materials}%{{{

There is no required textbook, but we will mainly use materials from the following books:

\begin{enumerate}
    
    \item \href{https://clauswilke.com/dataviz/}{Fundamentals of Data Visualization} by Claus O. Wilke (available online at \url{https://clauswilke.com/dataviz/})

\item \href{http://www.amazon.com/gp/product/0961392142}{The Visual Display of Quantitative Information (2nd ed.)} by E.R. Tufte: one of the foundational book on visualization. It contains a rich set of historical visualization, thoughtful discussion on visualization principles. 

\end{enumerate}

See also \href{https://yyiki.org/wiki/Data%20visualization/Books/}{Visualization books} and \href{https://yyiki.org/wiki/Data%20visualization/}{Data Visualization page} on my wiki. 


If you are still in the process of learning the basics of Python, the following books and websites may be helpful for you:

\begin{enumerate}%{{{

\item \url{https://docs.python.org/3/}: Python 3 Official Documentation

\item \url{http://www.diveintopython3.net/index.html}: Dive Into Python by Mark Pilgrim 

\item \url{http://www.learnpython.org}: A web-based interactive tutorial 

\item \url{https://github.com/ipython-books/minibook-2nd-code}: Learning IPython for Interactive Computing and Data Visualization by Cyrille Rossant: Introduction to IPython as well as lots of advanced analysis 

\end{enumerate}%}}}

If you are interested in web-based visualizations, you should check out the ObservableHQ and its tutorials for D3.js, Vega-Lite, and Observable Plot, all available at the following URL:

\begin{itemize}
    \item \url{https://observablehq.com/tutorials}
\end{itemize}

%}}}
\section{Final project}%{{{

See \url{https://github.com/yy/dviz-course/wiki/Projects} for the final project details, including the deliverables, types of projects, and some project ideas. 

%}}}
\section{Course Schedule}%{{{

The schedule may change. See also \href{https://registrar.indiana.edu/official-calendar/index.shtml}{IU Official Calendar}. 

\subsection{Key dates}\label{sub:key_dates} %{{{

Mark your calendar and plan ahead! 
%The summer course is fast-paced and end-heavy (both the exam and the final project are due in the last week of the course). So, be sure to start the project early and plan ahead. 

\begin{itemize}
%\item Project proposal due: \textbf{06/07}
\item Project proposal due: \textbf{10/03}
%\item (Residential) Project presentation: \textbf{12/4} and \textbf{12/6} 
%\item Exam: 7/22--7/26
\item Exam: \textbf{11/18} (online: \textbf{11/18}--\textbf{11/22})
\item Project presentation due: \textbf{12/09}
\item Project presentation (residential): \textbf{12/09} and \textbf{12/11}
\item Final paper due: \textbf{12/12}
\end{itemize} 

%}}}

\subsection{Schedule}\label{sub:schedule}%{{{

\begin{tabular}{@{}cll@{}} \toprule
  Week & Date   & Topic \\\midrule
  01 & 08/26--  & Why visualization? | Visualization tools \\
  02 & 09/02--  & Labor day | History and Integrity \\
  03 & 09/09--  & Perception \\
  04 & 09/16--  & Design principles \\
  05 & 09/23--  & Data types and tidy data \\
  06 & 09/30--  & Histogram and boxplot \\
  07 & 10/07--  & Estimation \\
  08 & 10/14--  & Logscale and beyond 1D \\
  09 & 10/21--  & High-dimensional data \\
  10 & 10/28--  & Maps \\
  11 & 11/04--  & Maps | Text \\
  12 & 11/11--  & Networks \\
  13 & 11/18--  & Exam \& Exam review\\
  14 & 11/25--  & Thanksgiving break \\
  15 & 12/02--  & Project hack week \\
  16 & 12/09--  & Project Presentation \\
  17 & 12/16--  & Final exam week (no exam) \\
  \bottomrule
\end{tabular}

%}}}

%\subsection{Schedule and Readings}\label{sub:schedule}%{{{

\subsubsection{Week 1 (5/6-): Why visualization?} %{{{

\begin{itemize}\itemsep0em 
\item J. Heer \emph{et al}. A Tour through the Visualization Zoo. \url{https://queue.acm.org/detail.cfm?id=1805128}
\item J. VanderPlas, The Python Visualization Landscape. \url{https://youtu.be/FytuB8nFHPQ}
\item Further readings: \url{https://github.com/yy/dviz-course/blob/master/m01-intro/class.md}
\end{itemize}	

%}}}
\subsubsection{Week 2 (5/13-): History and integrity}%{{{

\begin{itemize}\itemsep0em 
\item E.R. Tufte, The Visual Display of Quantitative Information, Ch.~1--2.
\item C.O. Wilke, Fundamentals of Data Visualization Ch.~1 (\url{https://serialmentor.com/dataviz/introduction.html}). 
\item Further readings: \url{https://github.com/yy/dviz-course/blob/master/m02-history/class.md} and \url{https://github.com/yy/dviz-course/blob/master/m03-integrity/class.md}
\end{itemize}	

%\subsubsection{Week 3 (5/20-): Labor day | Web, Declarative vs. Procedural visualization }

%}}}
\subsubsection{Week 3 (5/20-): Perception}%{{{

\begin{itemize}\itemsep0em 
\item C.G. Healey, Perception in Visualization, \url{https://www.csc2.ncsu.edu/faculty/healey/PP/index.html}
\item B. Wong, Color Coding, Nature Methods (2010).
\item B. Wong, Avoiding color, Nature Methods (2011). 
\item C.O. Wilke, Fundamentals of Data Visualization Ch.~4 Color scales (\url{https://serialmentor.com/dataviz/color-basics.html}). 
\item C.O. Wilke, Fundamentals of Data Visualization Ch.~15 Common pitfalls of color use (\url{https://serialmentor.com/dataviz/color-pitfalls.html}).
\item Further readings: \url{https://github.com/yy/dviz-course/blob/master/m04-perception/class.md}
\end{itemize}	
%}}}
\subsubsection{Week 4 (5/27-): Design }%{{{

\begin{itemize}\itemsep0em 
\item B. Wong, Gestalt Principles I \& II, Nature Methods (2010). 
\item E.R. Tufte, The Visual Display of Quantitative Information, Ch.~4.
\item S. Bateman et al., Useful Junk? The Effects of Visual Embellishment on Comprehension and Memorability of Charts, CHI'10.
\item C.O. Wilke, Fundamentals of Data Visualization Ch.~18--21 (\url{https://serialmentor.com/dataviz/optimize-data-signal.html}). 
\item Further readings: \url{https://github.com/yy/dviz-course/blob/master/m05-design/class.md}
\end{itemize}	
%}}}
\subsubsection{Week 5 (6/3-): Data Types and 1-D data } %{{{

\begin{itemize}\itemsep0em 
\item H. Wickham, Tidy Data, Journal of Statistical Software, \url{https://vita.had.co.nz/papers/tidy-data.pdf}
\item C.O. Wilke, Fundamentals of Data Visualization Ch.~14 (\url{https://serialmentor.com/dataviz/overlapping-points.html}). 
\item Further readings: \url{https://github.com/yy/dviz-course/blob/master/m06-data/class.md}
\end{itemize}	
%}}}
\subsubsection{Week 6 (6/10-): Histogram and Boxplot }%{{{

\begin{itemize}\itemsep0em 
\item C.O. Wilke, Fundamentals of Data Visualization Ch.~6--7 (\url{https://serialmentor.com/dataviz/overlapping-points.html}). 
\item Further readings: \url{https://github.com/yy/dviz-course/blob/master/m07-1D/class.md} and \url{https://github.com/yy/dviz-course/blob/master/m08-histogram/class.md}
\end{itemize}	
%}}}
\subsubsection{Week 7 (6/17-): Estimation and logscale }%{{{

\begin{itemize}\itemsep0em 
\item C.O. Wilke, Fundamentals of Data Visualization Ch.~8--9 (\url{https://serialmentor.com/dataviz/overlapping-points.html}). 
\item Further readings: \url{https://github.com/yy/dviz-course/blob/master/m09-estimation/class.md}
\item Khan Academy: Logarithmic scale with Vi Hart (\url{https://www.khanacademy.org/math/algebra2/exponential-and-logarithmic-functions/logarithmic-scale}). 
\end{itemize}	
%}}}
\subsubsection{Week 8 (6/24-): High-dimensional data }%{{{

\begin{itemize}\itemsep0em 
\item C.O. Wilke, Fundamentals of Data Visualization Ch.~11 (\url{https://serialmentor.com/dataviz/visualizing-associations.html}). 
\item 3Blue1Brown, Eigenvectors and eigenvalues \url{https://www.youtube.com/watch?v=PFDu9oVAE-g}. 
\item Victor Powell, PCA \url{http://setosa.io/ev/principal-component-analysis/}.
\item L. van der Maaten \& G. Hinton, Visualizing data using t-SNE, JMLR 2008 \url{http://www.jmlr.org/papers/volume9/vandermaaten08a/vandermaaten08a.pdf}.
\item Further readings: \url{https://github.com/yy/dviz-course/blob/master/m10-logscale/class.md} and \url{https://github.com/yy/dviz-course/blob/master/m11-highdim/class.md}
\end{itemize}	
%}}}
%\subsubsection{Week 9 (7/6-): High-dimensional data }%{{{

%\begin{itemize}\itemsep0em 
%\item C.O. Wilke, Fundamentals of Data Visualization Ch.~11 (\url{https://serialmentor.com/dataviz/visualizing-associations.html}). 
%\item 3Blue1Brown, Eigenvectors and eigenvalues \url{https://www.youtube.com/watch?v=PFDu9oVAE-g}. 
%\item Victor Powell, PCA \url{http://setosa.io/ev/principal-component-analysis/}.
%\item L. van der Maaten \& G. Hinton, Visualizing data using t-SNE, JMLR 2008 \url{http://www.jmlr.org/papers/volume9/vandermaaten08a/vandermaaten08a.pdf}.
%\end{itemize}	

%\subsubsection{Week 10 (7/8-): Mid-term}
%}}}
\subsubsection{Week 9 (7/1-): Maps }%{{{

\begin{itemize}\itemsep0em 
\item Vsauce, What does earth look like? \url{https://youtu.be/2lR7s1Y6Zig}
\item Vox, Why all world maps are wrong \url{https://youtu.be/kIID5FDi2JQ}
\item Further readings: \url{https://github.com/yy/dviz-course/blob/master/m12-maps/class.md}
\end{itemize}	
%}}}
\subsubsection{Week 10 (7/8-): Text and Networks } %{{{

\begin{itemize}\itemsep0em 
\item J. Harris, Word clouds considered harmful, \url{http://www.niemanlab.org/2011/10/word-clouds-considered-harmful/}. 
\item The Observatory of Economic Complexity, \url{https://atlas.media.mit.edu/en/profile/country/usa/}.
\item Further readings: \url{https://github.com/yy/dviz-course/blob/master/m13-text/class.md} and \url{https://github.com/yy/dviz-course/blob/master/m14-networks-and-interactive/class-network.md}
\end{itemize}	
%\subsubsection{Week 14 (11/19-): Thanksgiving break}
%\subsubsection{Week 15 (11/26-): Project Hacks}
%}}}
\subsubsection{Week 11 (7/15-): Project week} %{{{
%}}}
%\subsubsection{Week 14 (11/25-): Thanksgiving break} %{{{
%}}}
%\subsubsection{Week 15 (12/2-): Project Hack week}%{{{
%}}}
\subsubsection{Week 12 (7/22-): Final exam week}%{{{
%}}}

%}}}

%}}}

\section{Policies}%{{{
\begin{enumerate}%{{{
    \setlength\itemsep{1em}

%    \item \emph{Let's keep everyone safer together.} Don't be a jerk. Free masks are available near the entrance of every building. Carry some extra. I'll carry some extra as well. If you don't follow IU's mask policy (``all students, faculty, and staff should wear a mask that fully covers the wearer’s nose and mouth''), I will have to report you to the Office of Student Conduct and there can be sanctions. 

\item \emph{Be honest.} Don't be a cheater. Your assignments and papers should be your own work.  
If you find useful resources for your assignments, share them and cite them. 
If your friends helped you, acknowledge them. 
You should feel free to discuss both online and offline (except for the exam), but do not show your code directly.  
Any cases of academic misconduct (cheating, fabrication, plagiarism, etc) will be reported to the School and the Dean of Students, following the standard procedure. 
\emph{Cheating is not cool}. 

\item \emph{Missing classes.} If you were to miss a class, you need to notify the instructor and TAs \emph{before} the class begins to get an accommodation, except in extreme circumstances. You can then take the in-class quiz on the same day, ideally before the class ends. If your circumstances do not allow this arrangement, you have to get an explicit permission. 

\item \emph{You have the responsibility of backing up all your data and code}.
Always back up your code and data. You should at least use Google Drive or Dropbox at the minimum.
You can also use cloud services like Google Colaboratory.
Ideally, learn version control systems and use \url{https://github.iu.edu} or \url{https://github.com}. 
Loss of data, code, or papers (e.g.~due to malfunction of your laptop) is not an acceptable excuse for delayed or missing submission. 

\item \emph{Disabilities.} Every attempt will be made to accommodate qualified
students with disabilities (e.g.~mental health, learning, chronic health,
physical, hearing, vision, neurological, etc.). You must have established your
eligibility for support services through Disability Services for Students. Note
that services are confidential, may take time to put into place, and are not
retroactive.  Captions and alternate media for print materials may take three
or more weeks to get produced. Please contact Disability Services for Students
at \url{http://disabilityservices.indiana.edu} or 812-855-7578 as soon as
possible if accommodations are needed. The office is located on the third
floor, west tower, of the Wells Library (Room W302). Walk-ins are welcome 8 AM
to 5 PM, Monday through Friday. You can also locate a variety of campus
resources for students and visitors who need assistance at
\url{http://www.iu.edu/~ada/index.shtml}. 

\item \emph{Bias-based incidents.} Any act of discrimination or harassment based on 
race, ethnicity, religious affiliation, gender, gender identity, sexual orientation, or
disability can be reported to \texttt{biasincident@indiana.edu} or to the Dean of Students Office at (812) 855-8188. 

\item \emph{Sexual misconduct and Title IX.} 
Title IX and IU's Sexual Misconduct Policy prohibit sexual misconduct in any
form, including sexual harassment, sexual assault, stalking, and dating and
domestic violence.  If you have experienced sexual misconduct, or know someone
who has, you can use university resources:  

\begin{enumerate}
    
\item The Sexual Assault Crisis Services (SACS) at (812) 855-8900 (counseling services)
\item Confidential Victim Advocates (CVA) at (812) 856-2469 (advocacy and advice services)
\item IU Health Center at (812) 855-4011 (health and medical services)

\end{enumerate}

It is also important that you know that Title IX and University policy require me to share any information brought to my attention about potential sexual misconduct, with the campus Deputy Title IX Coordinator or IU's Title IX Coordinator. 
In that event, those individuals will work to ensure that appropriate measures are taken and resources are made available. 
Protecting student privacy is of utmost concern, and information will only be shared with those that need to know to ensure the University can respond and assist. 
Visit \emph{stopsexualviolence.iu.edu} to learn more. 

%\item \emph{Bring your laptop on Wednesdays}. However, \emph{no electronics---laptops, tablets, and smartphones---may be used in the class}, unless the usage is specifically requested by the instructors. 
%It has been shown that \href{http://www.scientificamerican.com/article/a-learning-secret-don-t-take-notes-with-a-laptop/}{using laptops in class hurts learning, \emph{even if} you are using it to take notes}.  
%If you must have electronics due to a special reason, please obtain a permission beforehand. 

%\item \emph{Inform your excused absences prior to class}. Please contact the instructor prior to the class that you cannot attend. 

%\item \emph{Late assignments}. There will be 10\% late penalty for the late assignments unless excused. 

\item If you have any mental health issues, don't hesitate to contact \href{http://healthcenter.indiana.edu/counseling/index.shtml}{IU's Counseling and Psychological Services}, which provides free counseling sessions. Also, please contact Disability Services for Students at \url{http://disabilityservices.indiana.edu} or 812-855-7578 as soon as possible if accommodations are needed. See ``Disabilities'' section for more information. 


\end{enumerate}%}}}
%}}}

\end{document} %}}}
%}}}
